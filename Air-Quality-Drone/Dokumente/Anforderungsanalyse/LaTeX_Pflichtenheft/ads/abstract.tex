%!TEX root = ../dokumentation.tex

\pagestyle{empty}

\iflang{de}{%
% Dieser deutsche Teil wird nur angezeigt, wenn die Sprache auf Deutsch eingestellt ist.
\renewcommand{\abstractname}{\langabstract} % Text für Überschrift

% \begin{otherlanguage}{english} % auskommentieren, wenn Abstract auf Deutsch sein soll
\begin{abstract}
Logfiles beinhalten eine große Menge an Daten, deren Analyse bei der Suche nach Fehlern und der Überwachung einer IT Infrastruktur eine große Hilfe sind. Dabei stellen sich mehrere Herausforderungen. Die Erste ist, dass Logfiles textbasiert sind. Der Nachteil hierbei ist, dass im Vergleich zu einer Datenbank oder einer XML Datei textbasierte Dateien keine klar auslesbare oder durchsuchbare Struktur besitzen. Die Zweite ist, dass Systeme so viele Informationen wie möglich loggen und dadurch die nützlichen bzw. wichtigen Informationen erst herausgefiltert werden müssen.

Für die Analyse von textbasierten Daten eignet sich sehr gut das MapReduce Modell. Außerdem lässt sich das Modell sehr einfach skalieren und auf mehrere Programmläufe verteilen (Master-Worker). Das Apache Hadoop Projekt stellt sowohl für MapReduce, als auch für die Verwaltung von mehreren Programmläufen, ein Basisframework bereit, mit welchem die Entwicklung eines Analyseprogramms durchgeführt werden soll.

Ziel dieser Arbeit ist die Entwicklung einer prototypischen Anwendung zur formatunabhängigen Analyse von Logfiles unter Zuhilfenahme von Apache Hadoop MapReduce.  Die Anwendung soll die bisher vorhandenen Monitoring Systeme innerhalb der Infrastruktur ergänzen, wodurch Informationen über den Zustand des Systems schneller erhoben werden können. Des Weiteren sollen aufkommende Fehler besser erkannt werden, um die Reaktionszeit auf diese zu optimieren.
\end{abstract}
% \end{otherlanguage} % auskommentieren, wenn Abstract auf Deutsch sein soll
}



\iflang{en}{%
% Dieser englische Teil wird nur angezeigt, wenn die Sprache auf Englisch eingestellt ist.
\renewcommand{\abstractname}{\langabstract} % Text für Überschrift

\begin{abstract}
An abstract is a brief summary of a research article, thesis, review, conference proceeding or any in-depth analysis of a particular subject or discipline, and is often used to help the reader quickly ascertain the paper's purpose. When used, an abstract always appears at the beginning of a manuscript, acting as the point-of-entry for any given scientific paper or patent application. Abstracting and indexing services for various academic disciplines are aimed at compiling a body of literature for that particular subject.

The terms précis or synopsis are used in some publications to refer to the same thing that other publications might call an ``abstract''. In ``management'' reports, an executive summary usually contains more information (and often more sensitive information) than the abstract does.

Quelle: \url{http://en.wikipedia.org/wiki/Abstract_(summary)}

\end{abstract}
}