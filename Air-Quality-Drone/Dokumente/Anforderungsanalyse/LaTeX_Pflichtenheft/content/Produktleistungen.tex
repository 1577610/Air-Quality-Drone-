%!TEX root = ../dokumentation.tex



\chapter{Produktleistungen}\label{cha:Produktleistungen}

\begin{tabular}{|>{\columncolor{lightgray}}p{3 cm}|p{13 cm}|}
	\hline
	\textbf{ID} & \textbf{<<L-010>>} \\
	\hline
	\textbf{Leistung} & Systemanforderungen \\
	\hline
	\textbf{Beschreibung} & Die App muss auf den folgenden Betriebssystemen lauffähig sein:
	\begin{itemize}
		\item iOS 9 und neuer		
	\end{itemize} \\
	\hline
\end{tabular}

\begin{tabular}{|>{\columncolor{lightgray}}p{3 cm}|p{13 cm}|}
	\hline
	\textbf{ID} & \textbf{<<L-020>>} \\
	\hline
	\textbf{Leistung} & GPS Standort ermitteln \\
	\hline
	\textbf{Beschreibung} & Die App muss über die Drohne den GPS Standort der Drohne ermitteln können. \\
	\hline
\end{tabular}

\begin{tabular}{|>{\columncolor{lightgray}}p{3 cm}|p{13 cm}|}
	\hline
	\textbf{ID} & \textbf{<<L-030>>} \\
	\hline
	\textbf{Leistung} & NO\textsubscript{x} Werte messen  \\
	\hline
	\textbf{Beschreibung} & Das Produkt muss NO\textsubscript{x} Werte in seiner Umgebung messen können. \\
	\hline
\end{tabular}

\begin{tabular}{|>{\columncolor{lightgray}}p{3 cm}|p{13 cm}|}
	\hline
	\textbf{ID} & \textbf{<<L-040>>} \\
	\hline
	\textbf{Leistung} & Feinstaub Werte messen (2,5 $\mu m$ \& 10 $\mu m$)  \\
	\hline
	\textbf{Beschreibung} & Das Produkt muss Feinstaub Werte in seiner Umgebung messen können. \\
	\hline
\end{tabular}

\begin{tabular}{|>{\columncolor{lightgray}}p{3 cm}|p{13 cm}|}
	\hline
	\textbf{ID} & \textbf{<<L-050>>} \\
	\hline
	\textbf{Leistung} & O\textsubscript{3} Werte messen  \\
	\hline
	\textbf{Beschreibung} & Das Produkt muss O\textsubscript{3} Werte in seiner Umgebung messen können.	 \\
	\hline
\end{tabular}

\begin{tabular}{|>{\columncolor{lightgray}}p{3 cm}|p{13 cm}|}
	\hline
	\textbf{ID} & \textbf{<<L-060>>} \\
	\hline
	\textbf{Leistung} & CO Werte messen  \\
	\hline
	\textbf{Beschreibung} & Das Produkt muss CO Werte in seiner Umgebung messen können.	 \\
	\hline
\end{tabular}

\begin{tabular}{|>{\columncolor{lightgray}}p{3 cm}|p{13 cm}|}
	\hline
	\textbf{ID} & \textbf{<<L-070>>} \\
	\hline
	\textbf{Leistung} & SO\textsubscript{2} Werte messen  \\
	\hline
	\textbf{Beschreibung} & Das Produkt muss SO\textsubscript{2} Werte in seiner Umgebung messen können.	 \\
	\hline
\end{tabular}

\begin{tabular}{|>{\columncolor{lightgray}}p{3 cm}|p{13 cm}|}
	\hline
	\textbf{ID} & \textbf{<<L-080>>} \\
	\hline
	\textbf{Leistung} & CO\textsubscript{2} Werte messen  \\
	\hline
	\textbf{Beschreibung} & Das Produkt muss CO\textsubscript{2} Werte in seiner Umgebung messen können.	 \\
	\hline
\end{tabular}

\begin{tabular}{|>{\columncolor{lightgray}}p{3 cm}|p{13 cm}|}
	\hline
	\textbf{ID} & \textbf{<<L-090>>} \\
	\hline
	\textbf{Leistung} & CH\textsubscript{4} Werte messen  \\
	\hline
	\textbf{Beschreibung} & Das Produkt muss CH\textsubscript{4} Werte in seiner Umgebung messen können.	 \\
	\hline
\end{tabular}

\begin{tabular}{|>{\columncolor{lightgray}}p{3 cm}|p{13 cm}|}
	\hline
	\textbf{ID} & \textbf{<<L-100>>} \\
	\hline
	\textbf{Leistung} & VOCs Werte messen  \\
	\hline
	\textbf{Beschreibung} & Das Produkt muss VOCs Werte in seiner Umgebung messen können.	 \\
	\hline
\end{tabular}

\begin{tabular}{|>{\columncolor{lightgray}}p{3 cm}|p{13 cm}|}
	\hline
	\textbf{ID} & \textbf{<<L-110>>} \\
	\hline
	\textbf{Leistung} & Luftfeuchtigkeit messen  \\
	\hline
	\textbf{Beschreibung} & Das Produkt muss die Luftfeuchtigkeit in seiner Umgebung messen können.	 \\
	\hline
\end{tabular}

\begin{tabular}{|>{\columncolor{lightgray}}p{3 cm}|p{13 cm}|}
	\hline
	\textbf{ID} & \textbf{<<L-120>>} \\
	\hline
	\textbf{Leistung} & Temperatur messen  \\
	\hline
	\textbf{Beschreibung} & Das Produkt muss die Temperatur in seiner Umgebung messen können.	 \\
	\hline
\end{tabular}

\begin{tabular}{|>{\columncolor{lightgray}}p{3 cm}|p{13 cm}|}
	\hline
	\textbf{ID} & \textbf{<<L-130>>} \\
	\hline
	\textbf{Leistung} & Luftdruck messen  \\
	\hline
	\textbf{Beschreibung} & Das Produkt muss den Luftdruck in seiner Umgebung messen können.	 \\
	\hline
\end{tabular}

\begin{tabular}{|>{\columncolor{lightgray}}p{3 cm}|p{13 cm}|}
	\hline
	\textbf{ID} & \textbf{<<L-140>>} \\
	\hline
	\textbf{Leistung} & Zeit messen  \\
	\hline
	\textbf{Beschreibung} & Das Produkt muss die Zeiten der Messungen messen können.	 \\
	\hline
\end{tabular}