%!TEX root = ../dokumentation.tex



\chapter{Produktdaten}\label{cha:Produktdaten}
 





\section{Non-persistente Daten}\label{sec:Non-persistente Daten}

\begin{tabular}{|>{\columncolor{lightgray}}p{3 cm}|p{13 cm}|}
	\hline
	\textbf{ID} & \textbf{<<D-001>>} \\
	\hline
	\textbf{Inhalt} & Video-Stream \\
	\hline
	\textbf{Bestandteile} & 
	\begin{itemize}
		\item Video-Stream der Kamera an der Drohne		
	\end{itemize} \\
	\hline
\end{tabular}

\section{Persistente Daten}\label{sec:Persistente Daten}

\begin{tabular}{|>{\columncolor{lightgray}}p{3 cm}|p{13 cm}|}
	\hline
	\textbf{ID} & \textbf{<<D-010>>} \\
	\hline
	\textbf{Inhalt} & Flugrouten-Koordinaten \\
	\hline
	\textbf{Bestandteile} & 
	\begin{itemize}
		\item Beschreibung der Flugroute
		\item GPS-Koordinaten der abzufliegenden Punkte			
	\end{itemize} \\
	\hline
\end{tabular}

\begin{tabular}{|>{\columncolor{lightgray}}p{3 cm}|p{13 cm}|}
	\hline
	\textbf{ID} & \textbf{<<D-020>>} \\
	\hline
	\textbf{Inhalt} & Messprofil \\
	\hline
	\textbf{Bestandteile} & 
	\begin{itemize}
		\item Beschreibung des Messprofils
		\item Genauigkeit der Messung (in Messungen/Zeiteinheit)
		\item Zu messende Werte (Feinstaub, NO\textsubscript{x}, CO\textsubscript{2}, ...)
	\end{itemize} \\
	\hline
\end{tabular}

\begin{tabular}{|>{\columncolor{lightgray}}p{3 cm}|p{13 cm}|}
	\hline
	\textbf{ID} & \textbf{<<D-030>>} \\
	\hline
	\textbf{Inhalt} & Messdaten \\
	\hline
	\textbf{Bestandteile} & 
	\begin{itemize}
		\item Flugkoordinaten
		\item Zeitstempel
		\item Temperatur
		\item Feuchtigkeit
		\item Druck
		\item Feinstaub-Partikel-Konzentration
		\item Stickoxid-Konzentration
		\item Kohlenstoffdioxid-Konzentration
		\item Ozon-Konzentration
		\item Methan-Konzentration			
	\end{itemize} 
	Jeweils ein Datensatz pro vorgegebener Zeiteinheit\\
	\hline
\end{tabular}