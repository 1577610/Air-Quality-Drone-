%!TEX root = ../dokumentation.tex



\chapter{Produktfunktionen}\label{cha:Produktfunktionen}



\begin{tabular}{|>{\columncolor{lightgray}}p{3 cm}|p{13 cm}|}
	\hline
	\textbf{ID} & \textbf{<<F-010>>} \\
	\hline
	\textbf{Funktion} & App starten \\
	\hline
	\textbf{Akteur} & Anwender \\
	\hline
	\textbf{Beschreibung} & Der Anwender muss die App auf seinem Endgerät (iPhone/iPad) starten können.\\
	\hline
\end{tabular}

\begin{tabular}{|>{\columncolor{lightgray}}p{3 cm}|p{13 cm}|}
	\hline
	\textbf{ID} & \textbf{<<F-100>>} \\
	\hline
	\textbf{Funktion} & Flugroute anlegen \\
	\hline
	\textbf{Akteur} & Anwender \\
	\hline
	\textbf{Beschreibung} & Der Anwender muss in der Lage sein eine Flugroute in der App anzulegen. Durch setzen von Punkten auf einer Karte, kann die Flugroute gesetzt werden. Eine weitere Möglichkeit ist die Flugroute manuell abzufliegen, sodass diese gespeichert wird.\\
	\hline
\end{tabular}

\begin{tabular}{|>{\columncolor{lightgray}}p{3 cm}|p{13 cm}|}
	\hline
	\textbf{ID} & \textbf{<<F-110>>} \\
	\hline
	\textbf{Funktion} & Flugroute bearbeiten \\
	\hline
	\textbf{Akteur} & Anwender \\
	\hline
	\textbf{Beschreibung} & Der Anwender muss in der Lage sein eine bestehende Flugroute zu bearbeiten.\\
	\hline
\end{tabular}

\begin{tabular}{|>{\columncolor{lightgray}}p{3 cm}|p{13 cm}|}
	\hline
	\textbf{ID} & \textbf{<<F-120>>} \\
	\hline
	\textbf{Funktion} & Flugroute löschen \\
	\hline
	\textbf{Akteur} & Anwender \\
	\hline
	\textbf{Beschreibung} & Der Anwender muss in der Lage sein eine bestehende Flugroute zu löschen.\\
	\hline
\end{tabular}

\begin{tabular}{|>{\columncolor{lightgray}}p{3 cm}|p{13 cm}|}
	\hline
	\textbf{ID} & \textbf{<<F-130>>} \\
	\hline
	\textbf{Funktion} & Flugroute auswählen \\
	\hline
	\textbf{Akteur} & Anwender \\
	\hline
	\textbf{Beschreibung} & Der Anwender muss in der App eine Flugroute auswählen können. Es kann eine Flugroute ausgewählt werden. Existiert noch keine Flugroute kann eine neue Flugroute erstellt werden (F-100).\\
	\hline
\end{tabular}





\begin{tabular}{|>{\columncolor{lightgray}}p{3 cm}|p{13 cm}|}
	\hline
	\textbf{ID} & \textbf{<<F-200>>} \\
	\hline
	\textbf{Funktion} & Messprofil erstellen \\
	\hline
	\textbf{Akteur} & Anwender \\
	\hline
	\textbf{Beschreibung} & Der Anwender muss ein neues Messprofil erstellen können. Beim Erstellen muss die Messhäufigkeit und die zu messenden Daten ausgewählt werden.\\
	\hline
\end{tabular}

\begin{tabular}{|>{\columncolor{lightgray}}p{3 cm}|p{13 cm}|}
	\hline
	\textbf{ID} & \textbf{<<F-210>>} \\
	\hline
	\textbf{Funktion} & Messprofil ändern \\
	\hline
	\textbf{Akteur} & Anwender \\
	\hline
	\textbf{Beschreibung} & Der Anwender muss ein von ihm erstelltes Messprofil ändern können.\\
	\hline
\end{tabular}

\begin{tabular}{|>{\columncolor{lightgray}}p{3 cm}|p{13 cm}|}
	\hline
	\textbf{ID} & \textbf{<<F-220>>} \\
	\hline
	\textbf{Funktion} & Messprofil löschen \\
	\hline
	\textbf{Akteur} & Anwender \\
	\hline
	\textbf{Beschreibung} & Der Anwender muss ein von ihm erstelltes Messprofil löschen können.\\
	\hline
\end{tabular}

\begin{tabular}{|>{\columncolor{lightgray}}p{3 cm}|p{13 cm}|}
	\hline
	\textbf{ID} & \textbf{<<F-230>>} \\
	\hline
	\textbf{Funktion} & Messprofil auswählen \\
	\hline
	\textbf{Akteur} & Anwender \\
	\hline
	\textbf{Beschreibung} & Der Anwender muss vor dem Starten einer Route ein Messprofil auswählen können.\\
	\hline
\end{tabular}





\begin{tabular}{|>{\columncolor{lightgray}}p{3 cm}|p{13 cm}|}
	\hline
	\textbf{ID} & \textbf{<<F-300>>} \\
	\hline
	\textbf{Funktion} & Drohnenflug starten \\
	\hline
	\textbf{Akteur} & Anwender \\
	\hline
	\textbf{Beschreibung} & Der Anwender muss den ausgewählten Drohnenflug (F-130) starten können. Vor dem Start muss ein Messprofil ausgewählt werden (F-230).\\
	\hline
\end{tabular}

\begin{tabular}{|>{\columncolor{lightgray}}p{3 cm}|p{13 cm}|}
	\hline
	\textbf{ID} & \textbf{<<F-310>>} \\
	\hline
	\textbf{Funktion} & Drohnenflug abbrechen \\
	\hline
	\textbf{Akteur} & Anwender \\
	\hline
	\textbf{Beschreibung} & Der Anwender muss einen Drohnenflug, den er gestartet hat (F-300) abbrechen können.\\
	\hline
\end{tabular}




\begin{tabular}{|>{\columncolor{lightgray}}p{3 cm}|p{13 cm}|}
	\hline
	\textbf{ID} & \textbf{<<F-400>>} \\
	\hline
	\textbf{Funktion} & Messdaten als Tabelle anzeigen \\
	\hline
	\textbf{Akteur} & Anwender \\
	\hline
	\textbf{Beschreibung} & Der Anwender muss sich die Messdaten in einer Tabelle anzeigen lassen können. \\
	\hline
\end{tabular}

\begin{tabular}{|>{\columncolor{lightgray}}p{3 cm}|p{13 cm}|}
	\hline
	\textbf{ID} & \textbf{<<F-410>>} \\
	\hline
	\textbf{Funktion} & Messdaten in Karte anzeigen \\
	\hline
	\textbf{Akteur} & Anwender \\
	\hline
	\textbf{Beschreibung} & Der Anwender muss sich die Messdaten in einer Karte anzeigen lassen können. \\
	\hline
\end{tabular}

\begin{tabular}{|>{\columncolor{lightgray}}p{3 cm}|p{13 cm}|}
	\hline
	\textbf{ID} & \textbf{<<F-420>>} \\
	\hline
	\textbf{Funktion} & Messdaten exportieren \\
	\hline
	\textbf{Akteur} & Anwender \\
	\hline
	\textbf{Beschreibung} & Der Anwender muss die gemessenen Daten als csv-Datei exportieren können. \\
	\hline
\end{tabular}

\begin{tabular}{|>{\columncolor{lightgray}}p{3 cm}|p{13 cm}|}
	\hline
	\textbf{ID} & \textbf{<<F-430>>} \\
	\hline
	\textbf{Funktion} & Messdaten an den Server übermitteln \\
	\hline
	\textbf{Akteur} & Anwender \\
	\hline
	\textbf{Beschreibung} & Der Anwender muss die gemessenen Daten als csv-Datei mittels der App an den Server übermitteln können. \\
	\hline
\end{tabular}
