%!TEX root = ../dokumentation.tex



\chapter{Testszenarien und Testfälle}\label{cha:Testszenarien und Testfälle}

\begin{tabular}{|>{\columncolor{lightgray}}p{3 cm}|p{13 cm}|}
	\hline
	\textbf{ID} & \textbf{<<TC-010>>} \\
	\hline
	\textbf{Beschreibung} & App starten  \\
	\hline
	\textbf{Vorbedingung} & App ist auf dem Endgerät installiert	 \\
	\hline
	\textbf{Testschritte} & 
	\begin{enumerate}
		\item Der Anwender drückt auf das App Icon auf dem Endgerät
	\end{enumerate} \\
	\hline
	\textbf{Zu erwartendes Ergebnis} & App wird geöffnet	 \\
	\hline
\end{tabular}

\begin{tabular}{|>{\columncolor{lightgray}}p{3 cm}|p{13 cm}|}
	\hline
	\textbf{ID} & \textbf{<<TC-100>>} \\
	\hline
	\textbf{Beschreibung} & Flugroute anlegen  \\
	\hline
	\textbf{Vorbedingung} & -	 \\
	\hline
	\textbf{Testschritte} & 
	\begin{enumerate}
		\item Der Anwender wählt verschiedene aufeinanderfolgende Punkte auf einer Karte aus
		\item Der Anwender speichert die Flugroute
	\end{enumerate} \\
	\hline
	\textbf{Zu erwartendes Ergebnis} & Die Flugroute wird angelegt und gespeichert	 \\
	\hline
\end{tabular}

\begin{tabular}{|>{\columncolor{lightgray}}p{3 cm}|p{13 cm}|}
	\hline
	\textbf{ID} & \textbf{<<TC-110>>} \\
	\hline
	\textbf{Beschreibung} & Flugroute bearbeiten  \\
	\hline
	\textbf{Vorbedingung} & Mindestens eine Flugroute existiert	 \\
	\hline
	\textbf{Testschritte} & 
	\begin{enumerate}
		\item Der Anwender wählt eine existierende Flugroute (TC-130)
		\item Der Anwender ändert Punkte der Flugroute
		\item Der Anwender speichert die Flugroute
	\end{enumerate} \\
	\hline
	\textbf{Zu erwartendes Ergebnis} & Die Flugroute wird geändert und gespeichert	 \\
	\hline
\end{tabular}

\begin{tabular}{|>{\columncolor{lightgray}}p{3 cm}|p{13 cm}|}
	\hline
	\textbf{ID} & \textbf{<<TC-120>>} \\
	\hline
	\textbf{Beschreibung} & Flugroute löschen  \\
	\hline
	\textbf{Vorbedingung} & Mindestens eine Flugroute existiert	 \\
	\hline
	\textbf{Testschritte} & 
	\begin{enumerate}
		\item Der Anwender wählt eine existierende Flugroute (TC-130)
		\item Der Anwender löscht die Flugroute
	\end{enumerate} \\
	\hline
	\textbf{Zu erwartendes Ergebnis} & Die Flugroute wird gelöscht	 \\
	\hline
\end{tabular}

\begin{tabular}{|>{\columncolor{lightgray}}p{3 cm}|p{13 cm}|}
	\hline
	\textbf{ID} & \textbf{<<TC-130>>} \\
	\hline
	\textbf{Beschreibung} & Flugroute auswählen  \\
	\hline
	\textbf{Vorbedingung} & Mindestens eine Flugroute existiert	 \\
	\hline
	\textbf{Testschritte} & 
	\begin{enumerate}
		\item Der Anwender wählt eine existierende Flugroute
	\end{enumerate} \\
	\hline
	\textbf{Zu erwartendes Ergebnis} & Die Flugroute ist ausgewählt	 \\
	\hline
\end{tabular}





\begin{tabular}{|>{\columncolor{lightgray}}p{3 cm}|p{13 cm}|}
	\hline
	\textbf{ID} & \textbf{<<TC-200>>} \\
	\hline
	\textbf{Beschreibung} & Messprofil erstellen  \\
	\hline
	\textbf{Vorbedingung} & -	 \\
	\hline
	\textbf{Testschritte} & 
	\begin{enumerate}
		\item Der Anwender wählt eine Messhäufigkeit aus 
		\item Der Anwender wählt die zu messenden Daten aus
		\item Der Anwender speichert das Messprofil
	\end{enumerate} \\
	\hline
	\textbf{Zu erwartendes Ergebnis} & Das Messprofil wird erstellt und gespeichert	 \\
	\hline
\end{tabular}

\begin{tabular}{|>{\columncolor{lightgray}}p{3 cm}|p{13 cm}|}
	\hline
	\textbf{ID} & \textbf{<<TC-210>>} \\
	\hline
	\textbf{Beschreibung} & Messprofil bearbeiten  \\
	\hline
	\textbf{Vorbedingung} & Mindestens ein Messprofil existiert	 \\
	\hline
	\textbf{Testschritte} & 
	\begin{enumerate}
		\item Der Anwender wählt ein existierendes Messprofil (TC-230)
		\item Der Anwender ändert die Messhäufigkeit oder die zu messenden Daten
		\item Der Anwender speichert das Messprofil
	\end{enumerate} \\
	\hline
	\textbf{Zu erwartendes Ergebnis} & Das Messprofil wird geändert und gespeichert	 \\
	\hline
\end{tabular}

\begin{tabular}{|>{\columncolor{lightgray}}p{3 cm}|p{13 cm}|}
	\hline
	\textbf{ID} & \textbf{<<TC-220>>} \\
	\hline
	\textbf{Beschreibung} & Messprofil löschen  \\
	\hline
	\textbf{Vorbedingung} & Mindestens ein Messprofil existiert	 \\
	\hline
	\textbf{Testschritte} & 
	\begin{enumerate}
		\item Der Anwender wählt ein existierendes Messprofil (TC-230)
		\item Der Anwender löscht das Messprofil
	\end{enumerate} \\
	\hline
	\textbf{Zu erwartendes Ergebnis} & Das Messprofil wird gelöscht	 \\
	\hline
\end{tabular}

\begin{tabular}{|>{\columncolor{lightgray}}p{3 cm}|p{13 cm}|}
	\hline
	\textbf{ID} & \textbf{<<TC-230>>} \\
	\hline
	\textbf{Beschreibung} & Messprofil auswählen  \\
	\hline
	\textbf{Vorbedingung} & Mindestens ein Messprofil existiert	 \\
	\hline
	\textbf{Testschritte} & 
	\begin{enumerate}
		\item Der Anwender wählt ein existierendes Messprofil
	\end{enumerate} \\
	\hline
	\textbf{Zu erwartendes Ergebnis} & Das Messprofil ist ausgewählt	 \\
	\hline
\end{tabular}




\begin{tabular}{|>{\columncolor{lightgray}}p{3 cm}|p{13 cm}|}
	\hline
	\textbf{ID} & \textbf{<<TC-300>>} \\
	\hline
	\textbf{Beschreibung} & Drohnenflug starten  \\
	\hline
	\textbf{Vorbedingung} & -	 \\
	\hline
	\textbf{Testschritte} & 
	\begin{enumerate}
		\item Der Anwender wählt den Menüpunkt "Flug starten" aus 
		\item Der Anwender wählt eine existierende Flugroute der Combobox "Flugrouten" aus 
		\item Der Anwender wählt ein existierendes Messprofil aus der Combobox "Messprofile" aus 
		\item Der Anwender startet den Flug durch drücken des Buttons „Flug starten“
	\end{enumerate} \\
	\hline
	\textbf{Zu erwartendes Ergebnis} & Die Drohne startet und fliegt die ausgewählte Flugroute ab. Messungen werden nach dem ausgewählten Messprofil durchgeführt. \\
	\hline
\end{tabular}

\begin{tabular}{|>{\columncolor{lightgray}}p{3 cm}|p{13 cm}|}
	\hline
	\textbf{ID} & \textbf{<<TC-310>>} \\
	\hline
	\textbf{Beschreibung} & Drohnenflug abbrechen \\
	\hline
	\textbf{Vorbedingung} & Drohnenflug ist im Gange \& Drohne in Reichweite der Steuerung	 \\
	\hline
	\textbf{Testschritte} & 
	\begin{enumerate}
		\item Der Anwender drückt auf den Button "Messung abbrechen"
		\item Der Anwender bestätigt die auftretende Warnmeldung
	\end{enumerate} \\
	\hline
	\textbf{Zu erwartendes Ergebnis} & Der Drohnenflug wird abgebrochen und die Drohne kehrt zum Startpunkt zurück	 \\
	\hline
\end{tabular}






\begin{tabular}{|>{\columncolor{lightgray}}p{3 cm}|p{13 cm}|}
	\hline
	\textbf{ID} & \textbf{<<TC-400>>} \\
	\hline
	\textbf{Beschreibung} & Messdaten als Tabelle anzeigen \\
	\hline
	\textbf{Vorbedingung} & -	 \\
	\hline
	\textbf{Testschritte} & 
	\begin{enumerate}
		\item Der Anwender wählt den Menüpunkt "Messdaten anzeigen" aus
		\item Der Anwender wählt den Unterpunkt "Tabelle" aus
		\item Der Anwender wählt einen Zeitraum aus, für den er alle Messwerte angezeigt bekommen möchte
	\end{enumerate} \\
	\hline
	\textbf{Zu erwartendes Ergebnis} & Alle Messdaten werden in einer Tabelle angezeigt	 \\
	\hline
\end{tabular}

\begin{tabular}{|>{\columncolor{lightgray}}p{3 cm}|p{13 cm}|}
	\hline
	\textbf{ID} & \textbf{<<TC-410>>} \\
	\hline
	\textbf{Beschreibung} & Messdaten in Karte anzeigen \\
	\hline
	\textbf{Vorbedingung} & Es gibt existierende Messdaten	 \\
	\hline
	\textbf{Testschritte} & 
	\begin{enumerate}
		\item Der Anwender wählt den Menüpunkt "Messdaten anzeigen" aus
		\item Der Anwender wählt den Unterpunkt "Karte" aus
		\item Der Anwender wählt einen Zeitraum aus, für den er alle Messwerte angezeigt bekommen möchte
	\end{enumerate} \\
	\hline
	\textbf{Zu erwartendes Ergebnis} & Alle Messdaten werden in einer Karte angezeigt	 \\
	\hline
\end{tabular}

\begin{tabular}{|>{\columncolor{lightgray}}p{3 cm}|p{13 cm}|}
	\hline
	\textbf{ID} & \textbf{<<TC-420>>} \\
	\hline
	\textbf{Beschreibung} & Messdaten exportieren \\
	\hline
	\textbf{Vorbedingung} & -	 \\
	\hline
	\textbf{Testschritte} & 
	\begin{enumerate}
		\item Der Anwender wählt den Menüpunkt "Messdaten exportieren" aus
		\item Der Anwender wählt den Speicherort aus
	\end{enumerate} \\
	\hline
	\textbf{Zu erwartendes Ergebnis} & Alle Messdaten werden als csv-Datei exportiert	 \\
	\hline
\end{tabular}

\begin{tabular}{|>{\columncolor{lightgray}}p{3 cm}|p{13 cm}|}
	\hline
	\textbf{ID} & \textbf{<<TC-430>>} \\
	\hline
	\textbf{Beschreibung} & Messdaten an den Server übermitteln \\
	\hline
	\textbf{Vorbedingung} & -	 \\
	\hline
	\textbf{Testschritte} & 
	\begin{enumerate}
		\item Der Anwender wählt den Menüpunkt "Messdaten an Server übermitteln" aus
	\end{enumerate} \\
	\hline
	\textbf{Zu erwartendes Ergebnis} & Alle Messdaten werden als csv-Datei an den Server übermittelt	 \\
	\hline
\end{tabular}