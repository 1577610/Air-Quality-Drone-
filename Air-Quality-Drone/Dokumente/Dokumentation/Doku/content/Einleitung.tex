%!TEX root = ../dokumentation.tex

\chapter{Einleitung}\label{cha:Einleitung}
In den lezten Jahren bekam das Thema der Luftqualität immer mehr Aufmerksamkeit und gewinnt immer mehr an Bedeutung in der Tagespolitik sowie in der Industrie. Hier ist vor allem die Automobilindustrie in den Fokus gerückt, da die Verbrennungsmotoren in einer sehr emotional geführten Debatte für einen Großteil der schlechten Luft in Großstädten verantwortlich gemacht werden. Nun trägt nicht nur der Verkehr sondern auch die Industrie mit verschiedenen Fabriken, wie auch andere Faktoren, wie zum Beispiel das heizen mit Holz im Winter, zur Verschlechterung der Luftqualität bei. Es wurden in den letzten Jahren immer mehr Messstationen in großen und kleineren Städten platziert, um die Luftqualität zu überwachen.   
\newline
Zum Thema Luftqualität stellen sich folgende Fragen, welche in der folgenden Arbeit teilweise beantwortet werden sollen.
\begin{itemize}
	\item Was ist Luftqualität?
	\item Kann man die Luftqualität messen?
	\item Was sind Faktoren für die Luftqualität?
	\item Was sind für den Menschen gefährliche Faktoren in der Luft?
\end{itemize}	
 
\section{Problemstellung}\label{sec:Problemstellung}
Von den im Kapitel Einleitung genannten Messtationen ist in der Region Stuttgart die Messstation am Neckartor die bekannteste. Diese misst die Luftqualität aber nur an einer Stelle. Hierbei kann man diskutieren, ob dieser Wert überhaupt aussagekräftig ist oder nicht. Es könnte sein, dass die Wahl für den Ort der Messtation missglückt ist und die gemessenen Werte deshalb nicht aussagekräftig sind. 
\newline
Ein weiterer Aspekt, welcher zu berücksichtigen ist, sind die Auswirkungen des Wetters auf die Luftqualität. 

\section{Aufgabenstellung}\label{sec:Aufgabenstellung}
Um die genannten Probleme zu umgehen, soll eine Air-Quality-Drone erstellt werden. Hierbei soll eine bereits existierende Drone mit Sensoren ausgestattet werden, welche klassische Werte zur Beurteilung der Luftqualität und zur Beurteilung der Umgebung, wie zum Beispiel die Luftfeuchtigkeit und Temperatur erfassen können. Eine Drone ist agil und kann an verschiedenen Orten und in verschiedenen Luftschichten Messungen durchführen. 
\newline
In dieser Arbeit soll ein Prototyp für eine Drohne zur Messung der Luftqualität erstellt werden. Zu der Drone soll eine App erstellt werden, über die die Drohne bedient werden kann. 