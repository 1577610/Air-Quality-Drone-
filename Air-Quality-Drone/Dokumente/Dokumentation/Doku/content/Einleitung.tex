%!TEX root = ../dokumentation.tex

\chapter{Einleitung}\label{cha:Einleitung}
In den lezten Jahren bekam das Thema der Luftqualität immer mehr Aufmerksamkeit und gewinnt immer mehr an Bedeutung in der Tagespolitik so wie in der Industrie. Hier ist vor allem die Autoindustrie in den Fokus gerückt da die Verbrennungsmotoren in einer sehr emotional geführten Debatte für einen Großteil der schlechten Luft in Großstädten verantwortlich gemacht werden. Nun trägt nicht nur der Verkehr sondern auch die Industrie mit verschiedenen Fabriken, wie auch andere Faktoren, wie zum Beispiel das heizen mit Holz im Winter zur Verschlechterung der Luftqualität bei.  
\newline
Hierbei stellen sich folgende Fragen, welche in der folgenden Arbeit beantwortet werden sollen.
\begin{itemize}
	\item Was ist Luftqualität?
	\item Kann man die Luftqualität messen?
	\item Was sind Faktoren für die Luftqualität?
	\item Was sind für den Menschen gefährliche Faktoren in der Luftqualität?
\end{itemize}	
 
 \section{Problemstellung}\label{sec:Problemstellung}
\section{Aufgabenstellung}\label{sec:Aufgabenstellung}