%!TEX root = ../dokumentation.tex

\chapter{Theoretische Grundlagen}\label{cha:Grundlagen}
Zu Beginn der Arbeit müssen die theoretischen Grundlagen angeeignet und recherchiert werden. Hierzu gehören das identifizieren und die Recherche verschiedener Faktoren der Luftqualität, wie auch das erstellen von Indexen zur Bewertung der Luftqualität. Für die Umsetzung ist der State of the Art der iOS-Appentwicklung zu recherchieren. Da die iOS-Programmierung hauptsächlich in SWIFT erfolgt, müssen die relevanten Eigenschaften der Apple eigenen Programmiersprache benutzt und angeeignet werden.

\section{Luftqualität}\label{sec:Luftqualität}

\subsection{Feinstaub}\label{subsec:Feinstaub}

\subsection{\acf{NOx}}\label{subsec:NOx}

\subsection{\acf{SOx}}\label{subsec:SOx}

\subsection{\acf{CO}}\label{subsec:CO}

\subsection{\acf{CO2}}\label{subsec:CO2}

\subsection{\acf{O3}}\label{subsec:O3}
	
\subsection{\acf{VOCs}}\label{subsec:VOCs}

\subsection{Luftqualität Indexe}\label{subsec:Luftqualität Indexe}

\section{iOS-Appentwicklung}\label{sec:ioS-Appentwicklung}

\subsection{Allgemeines Appentwicklung}\label{subsec:Allgemeines Appentwicklung}

\subsection{SWIFT}\label{subsec:SWIFT}
