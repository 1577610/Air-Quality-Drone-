%!TEX root = ../dokumentation.tex

\addchap{\langabkverz}
%nur verwendete Akronyme werden letztlich im Abkürzungsverzeichnis des Dokuments angezeigt
%Verwendung: 
%		\ac{Abk.}   --> fügt die Abkürzung ein, beim ersten Aufruf wird zusätzlich automatisch die ausgeschriebene Version davor eingefügt bzw. in einer Fußnote (hierfür muss in header.tex \usepackage[printonlyused,footnote]{acronym} stehen) dargestellt
%		\acs{Abk.}   -->  fügt die Abkürzung ein
%		\acf{Abk.}   --> fügt die Abkürzung UND die Erklärung ein
%		\acl{Abk.}   --> fügt nur die Erklärung ein
%		\acp{Abk.}  --> gibt Plural aus (angefügtes 's'); das zusätzliche 'p' funktioniert auch bei obigen Befehlen
%	siehe auch: http://golatex.de/wiki/%5Cacronym
%	
\begin{acronym}[YTMMM]
\setlength{\itemsep}{-\parsep}

\acro{ADC}{Analog-to-Digital Converter}
\acro{API}{Application Programming Interface}
\acro{AQI}{Air Quality Index}
\acro{CAD}{Computer-Aided Design}
\acro{CAQI}{Common Air Quality Index}
\acro{CO}{Kohlenstoffmonoxid}
\acro{CO2}{Kohlenstoffdioxid}
\acro{CSV}{Comma-Separated Values}
\acro{DJI}{Dà-Jiāng Innovations Science and Technology Co., Ltd}
\acro{EPA}{Environmental Protection Agency}
\acro{GPIO}{General Purpose Input/Output}
\acro{GPS}{Global Positioning System}
\acro{GUI}{Graphical User Interface}
\acro{IDE}{Integrated Development Environment}
\acro{IOT}{Internet of Things}
\acro{IP}{Internetprotokoll}
\acro{KB}{Kilobyte}
\acro{LTS}{Long Term Support}
\acro{MB}{Megabyte}
\acro{MCU}{Microcontroller Unit}
\acro{MVC}{Model View Controller}
\acro{NO2}{Stickstoffdioxid}
\acro{NOx}{Stickoxide}
\acro{O3}{Ozon}
\acro{OS}{Betriebssystem}
\acro{Pa}{Pascal}
\acro{PCB}{Printed Circuit Board}
\acro{PM}{Feinstaub}
\acro{SDK}{Software Development Kit}
\acro{SO2}{Schwefeldioxid}
\acro{SSID}{Service Set Identifier}
\acro{TCP}{Transmission Control Protocol}
\acro{UART}{Universal Asynchronous Receiver-Transmitter}
\acro{UDP}{User Datagram Protocol}
\acro{V}{Volt}
\acro{WLAN}{Wireless Local Area Network}
\acro{XDK}{Cross Domain Development Kit}














\end{acronym}
