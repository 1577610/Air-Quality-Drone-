%!TEX root = ../dokumentation.tex

\pagestyle{empty}

\iflang{de}{%
% Dieser deutsche Teil wird nur angezeigt, wenn die Sprache auf Deutsch eingestellt ist.
\renewcommand{\abstractname}{\langabstract} % Text für Überschrift

% \begin{otherlanguage}{english} % auskommentieren, wenn Abstract auf Deutsch sein soll
\begin{abstract}
Zur Luftqualitätsmessung wird derzeit von vielen Organisationen sowie vom deutschen Umweltbundesamt auf feste Messstationen gesetzt, welche kritische Kennwerte wie Feinstaub, Ozon und weitere messen sowie dokumentieren. Dabei stellt sich die Frage, ob es sinnvoll ist, die Luftqualität nur mittels einzelner Stichproben an immer den gleichen Orten zu bestimmen. \newline
Um dieses Problem zu umgehen, beschäftigt sich diese Arbeit mit der Entwicklung einer Drohne, an welcher Sensoren zur Messung der Luftqualität angebracht sind. Mithilfe dieser Drohne ist es nun möglich die Luftqualität nicht nur an verschiedenen Standorten flexibel zu messen, sondern auch die Einflüsse unterschiedlicher Höhen auf die Messwerte zu ermitteln.\newline
\newline
Für die Anbindung der Sensoren an die Drohne wird das Bosch \acf{XDK} verwendet, ein mit Sensoren ausgestatteter Mikrocontroller. Dieser sendet die erfassten Messdaten über ein \acf{WLAN} an die zur Drohnensteuerung angefertigte iOS-App.\newline
\newline
Ziel dieser Arbeit ist der Entwurf und die Umsetzung eines Messsystems, mittels dessen sich die Luftqualität mobil erfassen lässt. Dazu sollen neben den internen Sensoren des \acs{XDK} mehrere externe Sensoren mit diesem verbunden werden, sodass ein möglichst umfangreiches Abbild der Umgebung entsteht. Neben dem Entwurf der Schaltungen und deren Umsetzung wird im Rahmen dieser Arbeit auch ein Programmcode für das \acs{XDK} entwickelt, welcher dafür sorgt, dass die Sensoren korrekt ausgelesen und deren Daten an die App übermittelt werden.
\end{abstract}
% \end{otherlanguage} % auskommentieren, wenn Abstract auf Deutsch sein soll
}



\iflang{en}{%
% Dieser englische Teil wird nur angezeigt, wenn die Sprache auf Englisch eingestellt ist.
\renewcommand{\abstractname}{\langabstract} % Text für Überschrift

\begin{abstract}
An abstract is a brief summary of a research article, thesis, review, conference proceeding or any in-depth analysis of a particular subject or discipline, and is often used to help the reader quickly ascertain the paper's purpose. When used, an abstract always appears at the beginning of a manuscript, acting as the point-of-entry for any given scientific paper or patent application. Abstracting and indexing services for various academic disciplines are aimed at compiling a body of literature for that particular subject.

The terms précis or synopsis are used in some publications to refer to the same thing that other publications might call an ``abstract''. In ``management'' reports, an executive summary usually contains more information (and often more sensitive information) than the abstract does.

Quelle: \url{http://en.wikipedia.org/wiki/Abstract_(summary)}

\end{abstract}
}